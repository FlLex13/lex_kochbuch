\begin{recipe}{Gemüsesuppe}{Suppe}{1.5 h}{5}

\begin{tcolorbox}[colback=cookSecondary!60, colframe=white, left=4mm, right=4mm, boxrule=0pt, top=2mm,bottom=2mm, sharp corners]
\begin{minipage}{0.28\textwidth}
% \includegraphics[width=\linewidth]{src/images/spaghetti.jpg}
\end{minipage}\hfill
\begin{minipage}{0.68\textwidth}
\textit{Selbstgemachte Gemüsesuppe für die kalte Jahreszeit. Am besten mit Käspress-Knödel.}
\end{minipage}
\end{tcolorbox}

\vspace{6pt}

\begin{tcolorbox}[colback=white,colframe=black!5,boxrule=0.5pt,arc=3mm]
\textbf{Zutaten}
\begin{itemize}[leftmargin=*,noitemsep]
\item 1.5 rote Zwiebeln mit Schale
\item 1 Knoblauchzehe mit Schale
\item 3 Karotten (ungeschält)
\item 1 Stange Lauch
\item 1 Stück Knollensellerie
\item Frische Petersilie
\item Gewürze: 2 Lorbeerblätter, 6 Pfefferkörner, 6 Wacholderbeeren
\end{itemize}
\end{tcolorbox}

\vspace{4pt}

\begin{tcolorbox}[colback=white,colframe=black!5,boxrule=0.5pt,arc=3mm]
\textbf{Zubereitung}
\begin{itemize}[leftmargin=*,noitemsep]
\item Zwiebeln \& Knoblauch halbieren und in einem Topf ohne Öl an der Schnittseite braun rösten.
\item Das restliche Gemüse grob schneiden und mitrösten.
\item Gewürze dazugeben \& mit 1.2l Wasser ablöschen.
\item Nach 1h kochen das Gemüse absieben.
\end{itemize}
\end{tcolorbox}

\end{recipe}
\newpage