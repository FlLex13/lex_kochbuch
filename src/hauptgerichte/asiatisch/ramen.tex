\begin{recipe}{Tantanmen Ramen}{Asiatisch}{1 h}{2}

\begin{tcolorbox}[colback=cookSecondary!60, colframe=white, left=4mm, right=4mm, boxrule=0pt, top=2mm,bottom=2mm, sharp corners]
\begin{minipage}{0.28\textwidth}
\includegraphics[width=\linewidth]{src/images/ramen.jpg}
\end{minipage}\hfill
\begin{minipage}{0.68\textwidth}
\textit{Japanische Ramensuppe mit Hackfleisch.}
\end{minipage}
\end{tcolorbox}

\vspace{6pt}

\begin{tcolorbox}[colback=white,colframe=black!5,boxrule=0.5pt,arc=3mm]
\textbf{Zutaten}
\begin{itemize}[leftmargin=*,noitemsep]
\item Ramen Nudeln
\item Pak Choi
\item Frühlingszwiebeln
\item Mais
\item 4 Rameneier:
    \begin{itemize}
        \item 5 EL Sojasauce
        \item 2,5 EL Mirin
        \item 2,5 EL Wasser
        \item 1,5 TL Zucker
    \end{itemize}
\item Brühe:
    \begin{itemize}
        \item 500ml Sojamilch
        \item 300ml Gemüsebrühe
    \end{itemize}
\item Suppenbasis:
    \begin{itemize}
        \item 3 EL Sojasauce
        \item 4 EL Sesampaste (Tahin)
        \item 2 TL Reisessig
    \end{itemize}
\item Hackfleischmischung:
    \begin{itemize}
        \item 300g gemischtes Hack
        \item 1 gepresste Knoblauchzehe
        \item 1 kl. St. Ingwer (gepresst)
        \item 2 EL Sojasauce
        \item 2 EL Chili Bohnen Sauce od. Gochujang
    \end{itemize}
\end{itemize}
\end{tcolorbox}

\vspace{4pt}

\begin{tcolorbox}[colback=white,colframe=black!5,boxrule=0.5pt,arc=3mm]
\textbf{Zubereitung}
\begin{itemize}[leftmargin=*,noitemsep]
\item Eier 7min Kochen, abschrecken \& schälen.
\item Die Zutaten für die Rameneier miteinander vermischen, Eier einlegen \& in den Kühlschrank geben.
\item Brühe \& Milch langsam erwärmen und bis zuletzt auf dem Herd lassen.
\item Suppenbasis zusammen mixen und auf zwei Schüsseln aufteilen.
\item Hackfleisch anbraten, Knoblauch \& Ingwer und die restlichen Gewürze dazugeben. Zum Schluss evtl. noch Frühlingszwiebeln untermischen.
\item Ramen Nudeln kochen \& Pak Choi dämpfen.
\item Brühe vom Herd nehmen \& kleine Menge in die Suppenbasis einrühren. Danach den Rest dazugeben.
\item Nudeln in die Schüsseln geben. Mit Hackfleisch \& den restlichen Toppings anrichten
\end{itemize}
\end{tcolorbox}

\end{recipe}
\newpage