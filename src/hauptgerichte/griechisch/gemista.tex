\begin{recipe}{Gemista / gefüllte Paprika}{Reis}{1 h}{4}

\begin{tcolorbox}[colback=cookSecondary!60, colframe=white, left=4mm, right=4mm, boxrule=0pt, top=2mm,bottom=2mm, sharp corners]
\begin{minipage}{0.28\textwidth}
% \includegraphics[width=\linewidth]{src/images/spaghetti.jpg}
\end{minipage}\hfill
\begin{minipage}{0.68\textwidth}
\textit{Mit Hackfleisch gefüllte Paprika. Als Beilage passt Reis oder Kartoffeln.}
\end{minipage}
\end{tcolorbox}

\vspace{6pt}

\begin{tcolorbox}[colback=white,colframe=black!5,boxrule=0.5pt,arc=3mm]
\textbf{Zutaten}
\begin{itemize}[leftmargin=*,noitemsep]
\item 4 Paprika
\item 4 Kartoffeln
\item 2 gro{\ss}e Tomaten
\item 400 ml Gemü{\ss}ebrühe
\item 400 g Hackfleisch
\item 150 g Reis
\item 1 Zwiebel, gewürfelt
\item 2 Knoblauchzehen
\item 1 gro{\ss}e Tomate, geviertelt
\item 2 EL Tomatenmark
\item frische Petersilie \& Dill
\item Gewürze: Salz, Pfeffer, Paprika, Chilli, Gyros-Gewürz
\end{itemize}
\end{tcolorbox}

\vspace{4pt}

\begin{tcolorbox}[colback=white,colframe=black!5,boxrule=0.5pt,arc=3mm]
\textbf{Zubereitung}
\begin{itemize}[leftmargin=*,noitemsep]
\item Deckel von Paprika \& Tomaten abschneiden und aushöhlen. Tomateninhalt aufbewahren.
\item Hackfleisch in Olivenöl anrösten. Zwiebeln \& Knoblauch hinzugeben.
\item Danach Tomatenmark \& Reis dazugeben und kurz mitbraten \& würzen.
\item Tomateninhalt \& Tomatenviertel dazu \& 5 min köcheln lassen.
\item Vom Herd nehmen \& Petersilie und Dill unterrühren.
\item Paprika \& Tomaten mit der Hackfleisch-Reis-Mischung füllen und in eine Auflaufform stellen.
\item Kartoffeln schälen \& in Spalten schneiden. Rund um die Paprika in die Form geben.
\item Mit Gemü{\ss}ebrühe aufgie{\ss}en \& Olivenöl darüber träufeln.
\item Bei Umluft 180 \textdegree C ca. 50 min abgedeckt, anschlie{\ss}end 15 min offen backen.
\end{itemize}
\end{tcolorbox}

\end{recipe}
\newpage