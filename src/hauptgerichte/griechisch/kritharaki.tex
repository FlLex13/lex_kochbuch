\begin{recipe}{Kritharakit - Hackfleisch Auflauf}{Reis}{30 min}{4}

\begin{tcolorbox}[colback=cookSecondary!60, colframe=white, left=4mm, right=4mm, boxrule=0pt, top=2mm,bottom=2mm, sharp corners]
\begin{minipage}{0.28\textwidth}
% \includegraphics[width=\linewidth]{src/images/spaghetti.jpg}
\end{minipage}\hfill
\begin{minipage}{0.68\textwidth}
\textit{Hackfleischauflauf mit griechischen Nudeln \& Feta.}
\end{minipage}
\end{tcolorbox}

\vspace{6pt}

\begin{tcolorbox}[colback=white,colframe=black!5,boxrule=0.5pt,arc=3mm]
\textbf{Zutaten}
\begin{itemize}[leftmargin=*,noitemsep]
\item 250 g Kritharaki (griechische Nudeln)
\item 200 g Feta
\item 500 g gemischtes Hackfleisch
\item 2 Dosen passierte Tomaten (ca. 850 ml)
\item 1 EL Tomatenmark
\item 275 ml Gemüsebrühe
\item 100 ml Milch
\item 2 EL Creme Fraiche
\item 0.5 Zwiebel
\item 1 Knoblauchzehe
\item Gewürze: Salz, Pfeffer, Paprika, Thymian, Zimt, Cayennepfeffer
\end{itemize}
\end{tcolorbox}

\vspace{4pt}

\begin{tcolorbox}[colback=white,colframe=black!5,boxrule=0.5pt,arc=3mm]
\textbf{Zubereitung}
\begin{itemize}[leftmargin=*,noitemsep]
\item Hackfleisch in Öl anbraten. Gewürfelte Zwiebeln \& Knoblauch hinzugeben.
\item Mit Tomatenmark und Gewürzen kurz mitrösten.
\item Anschlie{\ss}end mit passierten Tomaten \& Brühe ablöschen und aufkochen lassen.
\item Milch \& Creme Fraiche unterrühren.
\item Kritharaki (ungekocht) hinzufügen und in Auflaufform füllen. Feta darüber bröseln.
\item Bei Umluft 150 \textdegree C ca. 30 min backen.
\item Auflauf herausnehmen und mit etwas Olivenöl \& Thymian darüber geben.
\end{itemize}
\end{tcolorbox}

\end{recipe}
\newpage