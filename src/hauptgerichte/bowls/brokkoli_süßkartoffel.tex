\begin{recipe}{Brokkoli Sü{\ss}kartoffel Bowl}{Bowl}{1 h}{2}

\begin{tcolorbox}[colback=cookSecondary!60, colframe=white, left=4mm, right=4mm, boxrule=0pt, top=2mm,bottom=2mm, sharp corners]
\begin{minipage}{0.28\textwidth}
% \includegraphics[width=\linewidth]{src/images/spaghetti.jpg}
\end{minipage}\hfill
\begin{minipage}{0.68\textwidth}
\textit{Ein Klassiker aus der R\&S Kantine. Nix für den Erbse!}
\end{minipage}
\end{tcolorbox}

\vspace{6pt}

\begin{tcolorbox}[colback=white,colframe=black!5,boxrule=0.5pt,arc=3mm]
\textbf{Zutaten}
\begin{itemize}[leftmargin=*,noitemsep]
\item 1 Tasse Reis
\item 3 faustgro{\ss}e Sü{\ss}kartoffeln
\item 1 Brokkoli
\item ca. 150g Ziegenfrischkäse
\item Granatapfelkerne
\end{itemize}
\end{tcolorbox}

\vspace{4pt}

\begin{tcolorbox}[colback=white,colframe=black!5,boxrule=0.5pt,arc=3mm]
\textbf{Zubereitung}
\begin{itemize}[leftmargin=*,noitemsep]
\item Sü{\ss}kartoffeln vierteln und bei 180°C ca. 45 min in den Backofen.
\item Die letzten 10 min die Brokkoliröschen dazugeben (Der Strunk kann geschält und in Scheiben geschnitten auch verwendet werden).
\item Reis kochen.
\item Reis, Sü{\ss}kartoffeln, Brokkoli in eine Bowl geben.
\item Ziegenfrischkäse \& Granatapfelkerne als Topping hinzufügen.
\end{itemize}
\end{tcolorbox}

\end{recipe}
\newpage