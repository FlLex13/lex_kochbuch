\begin{recipe}{Bunter Reis mit Thunfisch}{Reis}{20 Min}{2}

\begin{tcolorbox}[colback=cookSecondary!60, colframe=white, left=4mm, right=4mm, boxrule=0pt, top=2mm,bottom=2mm, sharp corners]
\begin{minipage}{0.28\textwidth}
% \includegraphics[width=\linewidth]{src/images/spaghetti.jpg}
\end{minipage}\hfill
\begin{minipage}{0.68\textwidth}
\textit{Tomatenreis mit Thunfisch und frischem Gemüse.}
\end{minipage}
\end{tcolorbox}

\vspace{6pt}

\begin{tcolorbox}[colback=white,colframe=black!5,boxrule=0.5pt,arc=3mm]
\textbf{Zutaten}
\begin{itemize}[leftmargin=*,noitemsep]
\item 200 g Reis
\item 1 Dose Thunfisch
\item 1 Dose passierte Tomaten (400 g)
\item 1 kleine Dose Mais
\item 1 kleine Dose Erbsen
\item 1 Zwiebel
\item 20 g Butter
\item Gemüsebrühe
\item Petersilie
\end{itemize}
\end{tcolorbox}

\vspace{4pt}

\begin{tcolorbox}[colback=white,colframe=black!5,boxrule=0.5pt,arc=3mm]
\textbf{Zubereitung}
\begin{itemize}[leftmargin=*,noitemsep]
\item Zwiebel in Würfel hacken und mit Reis in hei{\ss}er Butter glasig anschwitzen.
\item Einen Messbecher mit dem Gemüsesaft aus der Maisdose \& Wasser auf 400ml auffüllen.
\item Gemüsesaft \& passierte Tomaten zum Reis geben.
\item Bei kleiner Hitze köcheln lassen.
\item Erbsen \& Mais dazugeben und abschmecken.
\item Thunfisch hinzufügen und mit gehackter Petersilie bestreuen.
\item Mit Parmesan \& Pfeffer genie{\ss}en.
\end{itemize}
\end{tcolorbox}

\end{recipe}
\newpage