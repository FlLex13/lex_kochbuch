\begin{recipe}{Zucchinikuchen}{Kuchen}{1 h}{8}

\begin{tcolorbox}[colback=cookSecondary!60, colframe=white, left=4mm, right=4mm, boxrule=0pt, top=2mm,bottom=2mm, sharp corners]
\begin{minipage}{0.28\textwidth}
% \includegraphics[width=\linewidth]{src/images/spaghetti.jpg}
\end{minipage}\hfill
\begin{minipage}{0.68\textwidth}
\textit{Saftiger Zucchinikuchen mit Nüssen.}
\end{minipage}
\end{tcolorbox}

\vspace{6pt}

\begin{tcolorbox}[colback=white,colframe=black!5,boxrule=0.5pt,arc=3mm]
\textbf{Zutaten}
\begin{itemize}[leftmargin=*,noitemsep]
\item 250g geriebene Zucchini
\item 3 Eier
\item 200g Zucker
\item 200g Mehl
\item 150ml Öl
\item 100g gemahlene Haselnüsse
\item 50g Schokoraspel
\item 1 El Kakaopulver
\item 1 Tl Zimt
\item 1 Tl Backpulver
\item 1 Tl Natron
\end{itemize}
\end{tcolorbox}

\vspace{4pt}

\begin{tcolorbox}[colback=white,colframe=black!5,boxrule=0.5pt,arc=3mm]
\textbf{Zubereitung}
\begin{itemize}[leftmargin=*,noitemsep]
\item Alle Zutaten au{\ss}er Backpulver, Mehl \& Natron miteinander vermischen.
\item Im Anschluss Backpulver mit Mehl vermengen \& unter die Masse heben.
\item Abschlie{\ss}end Natron dazugeben \& Masse in eine gefettet Form geben.
\item Für ca. 45min bei \ang{180}C backen.
\end{itemize}
\end{tcolorbox}

\end{recipe}
\newpage