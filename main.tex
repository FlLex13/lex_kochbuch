% main.txt -- Hauptdatei (Präambel, Makros, Layout, Einbindungen)
\documentclass[11pt,a4paper]{article}
\usepackage[utf8]{inputenc}   % UTF-8 Eingabe
\usepackage[T1]{fontenc}
\usepackage[ngerman]{babel}   % Deutsche Trennregeln, Überschriften etc.
\usepackage{graphicx}         % Bilder
\usepackage{enumitem}         % Listenanpassung
\usepackage{hyperref}         % Links, Inhaltsverzeichnis
\usepackage{geometry}
\geometry{margin=2.5cm}

% Pfad für Bilder (z.B. src/images/)
\graphicspath{{src/images/}}

% Einfache Kopf-/Fußzeile
\usepackage{fancyhdr}
\pagestyle{fancy}
\fancyhf{}
\lhead{Mein Kochbuch}
\rhead{\leftmark}
\cfoot{\thepage}

% Ein paar nützliche Makros
\newcommand{\recipeAuthor}[1]{\textit{Autor: #1}}
\newcommand{\recipeTime}[1]{\textbf{Zubereitungszeit:} #1\\}
\newcommand{\recipePortions}[1]{\textbf{Portionen:} #1\\}
\newcommand{\recipeCategory}[1]{\textbf{Kategorie:} #1\\}

% Rezepte-Umgebung: erzeugt Abschnittsüberschrift + Metadaten
\usepackage{xparse}
\NewDocumentEnvironment{recipe}{mmmm}{%
  \section{#1} % Titel
  \recipeCategory{#2}
  \recipeTime{#3}
  \recipePortions{#4}
  \medskip
}{%
  \bigskip
}

\begin{document}

\begin{titlepage}
\thispagestyle{empty}

\vspace*{\fill}
\noindent\makebox[\linewidth][c]{%
\begin{tcolorbox}[enhanced,
colback=white,
colframe=cookPrimary,
boxrule=5pt,
arc=8mm,
left=8mm,right=8mm,top=8mm,bottom=8mm,
width=0.95\linewidth, % 95% der Textbreite
height=\dimexpr\textheight-4cm\relax, % Höhe innerhalb Textbereich
valign=center,center upper]
{\Huge\bfseries Ein Kochbuch}
\newline
{\Large für Lex Frauen \& Männer}
\end{tcolorbox}
}

\vspace*{\fill}

\clearpage
\setcounter{page}{1}
\pagestyle{fancy}
\end{titlepage}

\tableofcontents
\bigskip

\newpage

% Einbinden der Rezepte aus dem Ordner src


\section{Hauptgerichte}
\newpage

\input{src/hauptgerichte/italienisch/italienisch.tex}



% Weitere Rezepte: einfach neue .tex-Dateien in src/ legen und hier \input eintragen.

\end{document}